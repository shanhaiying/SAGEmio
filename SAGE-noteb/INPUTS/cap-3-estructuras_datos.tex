
Muchos de los programas que elaboramos sirven para transformar y analizar
datos que creamos o recibimos. En este resumen se revisan  las
estructuras de
datos que llamamos {\tt lista, tupla, cadena de caracteres, conjunto y
diccionario.} Una
característica común a todos estos tipos es que son {\sc iterables,} es decir,
\label{iterable}
que para
todos ellos tiene
sentido hacer un bucle \lstinline|for|  como
\begin{lstlisting}[numbers=none]
for item in <estructura_de_datos>:
	........
\end{lstlisting}

Nos referimos a estas  estructura de datos con el nombre gen\'erico de
{\itshape contenedor}, o m\'as precisamente {\itshape contenedor iterable.}






Otra cualidad común de los contenedores iterables  es la de poder comprobar
f\'acilmente si un
dato está o
no en ellas. Basta utilizar la partícula \lstinline|in|, de manera que
\begin{lstlisting}
dato in <estructura_de_datos>
\end{lstlisting} 
adquiere valor \lstinline|True| o \lstinline|False|.

Por último, la función \lstinline|len()| nos informa del número de datos en
cualquiera de  estas estructuras. 


\section{Datos b\'asicos: tipos}

Algunos lenguajes de programaci\'on, por ejemplo $C$, obligan a declarar
expl\'{\i}citamente de qu\'e tipo es cada variable. As\'{\i} por ejemplo, en $C$
hay tipos como
\begin{enumerate}
 \item {\tt long} para almacenar enteros, que deben pertenecer  al
intervalo\footnote{Este rango depende del compilador de $C$ utilizado, pero el
indicado es el habitual.} \[[-2147483648,2147483648]\].
\label{tipos}
Cada entero, de tipo {\tt long},  utiliza $32$ bits en la memoria del
ordenador. 

\item{\tt float} para n\'umeros decimales, de hasta $8$ cifras decimales,
pertenecientes al intervalo \[[1.175494351\cdot 10^{-38},3.402823466\cdot
10^{38}].\]

Cada decimal, de tipo {\tt float},  utiliza $32$ bits en la memoria del
ordenador. 

\item{\tt double} para n\'umeros decimales de mayor precisi\'on, dieciseis
cifras decimales,  que los dados por {\tt float}, concretamente, pertenecientes
al intervalo
\[[2.2250738585072014\cdot 10^{-308},1.7976931348623158\cdot 10^{308}].\]

Cada decimal, de tipo {\tt double},  utiliza $64$ bits en la memoria del
ordenador. 
\item {\tt char} en los que se almacenan los caracteres del c\'odigo ASCII, las
letras y s\'{\i}mbolos de puntuaci\'on.

Cada car\'acter, tipo {\tt char}, utiliza $8$ bits en la memoria. 
\end{enumerate}


La declaraci\'on de tipos es importante porque hace posible la compilaci\'on
del programa, es decir la conversi\'on del c\'odigo en un ejecutable que puede
funcionar de manera totalmente independiente.  Para ese funcionamiento
independiente, en otro ordenador en el que no est\'an necesariamente  ni el
c\'odigo ni el compilador de $C$, es esencial que el ejecutable contenga toda
la informaci\'on sobre la gesti\'on de la memoria RAM, que es lo que explica la
necesidad de la declaraci\'on de tipos.  El compilador traduce el c\'odigo $C$ a
c\'odigo m\'aquina del procesador que tenga el ordenador que usamos, de forma
que el ejecutable funcionar\'a en cualquier otro ordenador con ese procesador. 


En {\sage}, y en Python, no es necesario declarar los tipos de las variables
debido a que el c\'odigo no se compila, sino que se {\itshape interpreta}: el
c\'odigo se va traduciendo, por el {\itshape int\'erprete de Python},  a
lenguaje m\'aquina {\itshape sobre la marcha}, y es el int\'erprete el que se
ocupa de la gesti\'on de la memoria RAM. La ejecuci\'on de c\'odigo en
lenguajes interpretados es, en general, mucho m\'as lenta que en lenguajes
compilados, pero en cambio es mucho m\'as c\'omodo programar en Python ({\sage})
que en $C$.


En {\sage} los datos tienen tipos, aunque no hay que declararlos, y se pueden
producir {\itshape errores de tipo (Type error)}. Por ejemplo, si intentamos
factorizar un n\'umero decimal, se producir\'a un tal error.  





Los objetos b\'asicos en {\sage}, a partir de los que se construyen objetos
complejos, son similares a los de $C$: enteros, decimales, caracteres. Dado el
enfoque de {\sage} hacia las matem\'aticas, hay otros tipos de datos, n\'umeros
racionales o complejos, que tambi\'en podemos considerar b\'asicos.

Cuando nos encontramos con un error de tipo, al aplicar una funci\'on de
{\sage} a un cierto objeto $A$, podemos evaluar \lstinline|A.type()| o
\lstinline|type(A)|, lo que nos informar\'a del tipo que tiene el objeto $A$.
Analizando la documentaci\'on sobre la funci\'on podremos darnos cuenta de que
esa funci\'on concreta no se puede aplicar a los objetos del tipo de $A$, y
quiz\'a deberemos cambiar la definici\'on de $A$. 

\label{enteros}

Hay dos tipos diferentes de enteros: los enteros de Python y los enteros de
{\sage}. La \'unica diferencia, en la pr\'actica,  es que no es posible aplicar
a los enteros de Python los m\'etodos disponibles para los enteros de {\sage}.
Por ejemplo, los enteros de {\sage} tienen el m\'etodo \lstinline|.digits()|,
que nos devuelve una lista ordenada de los d\'{\i}gitos del entero, mientras que
si intentamos aplicar este m\'etodo a un entero de Python obtenemos un error de
tipo.  (Si usamos srange, nos da enteros de sage, range nos da enteros de piton)

En este
cap\'{\i}tulo veremos la forma de crear estructuras de datos complejas, que
contienen objetos b\'asicos de alguno de los tres tipos, o bien otras
estructuras de datos.  As\'{\i}, por ejemplo, podemos pensar una matriz como
una lista que contiene listas de la misma longitud,  que ser\'an las filas de la
matriz. Estas construcciones tienen, \hyperref[unhash]{como veremos}, algunas
restricciones, y as\'{\i}, por ejemplo, no podemos crear un conjunto cuyos
elementos sean listas. 




\section{Listas}

En este curso {\sc usaremos sobre todo listas y matrices} como estructuras de
datos, y las otras estructuras de datos servir\'an sobre todo en situaciones
particulares en las que necesitaremos un contenedor m\'as especializado.


\begin{enumerate}
 \item Definimos una lista separando sus elementos por comas y delimitando el
principio y
el final de la lista con corchetes cuadrados:
\begin{lstlisting}[numbers=none]
L = [1,2,3,4,5]
type(L), len(L), L
\end{lstlisting}
\begin{Output}
	(<type 'list'>, 5, [1, 2, 3, 4, 5])
\end{Output}

El nombre \lstinline|L| queda asignado a la lista \lstinline|[1,2,3,4,5]|. 
\item Podemos cambiar, en una lista $L$ existente, el elemento con \'{\i}ndice
$i$ reasignando su valor mediante \lstinline|L[i]=\dots|. Por ejemplo
\lstinline|L[1]=7| transforma la lista
$L$ del apartado anterior en $[1,7,3,4,5]$.
\item Usaremos la notación \emph{slice} 
para el acceso a sublistas. Si \lstinline|L| es una lista que ya ha sido
definida, 
\begin{itemize}
 \item seleccionamos el elemento de índice {\tt j} (o lugar {\tt j+1}) en
\lstinline|L|
con \lstinline|L[j]|;\footnote{Esto se debe a que {\tt
Sage}, y {\tt Python},  llaman elemento {\tt 0} de las listas al primero.}

\item seleccionamos con \lstinline|L[j:k:d]|, la sublista formada por los
elementos de
índices $j,j+d,j+2d,..., j+\ell d<k$. Si~no se
especifica el tercer argumento, los \emph{saltos} son de uno en uno. La ausencia
de
alguno de los dos primeros argumentos, manteniendo los dos
puntos de separación, indica que su valor es el más extremo (primero o último).
\begin{lstlisting}
L=[0,1,2,3,4,5,6,7,8,9,10,11,12]
print (L[3], L[1:5:2], L[6:10], L[11:], L[:4], L[::3])
\end{lstlisting}
\begin{Output}
	(3, [1, 3], [6, 7, 8, 9], [11, 12], [0, 1, 2, 3], [0, 3, 6, 9, 12])
\end{Output}
\end{itemize}
\item El signo \lstinline|+| concatena listas
\begin{lstlisting}
[1,3,5]+[2,4,6] 
\end{lstlisting}
\begin{Output}
	[1, 3, 5, 2, 4, 6]
\end{Output}

\item Se abrevia la concatenación  \lstinline|k| veces de una misma lista
\lstinline|L|
con \lstinline|k*L|. 


\end{enumerate}
\subsection{Métodos y funciones con listas}

\begin{enumerate}

\item Al aplicar un método a un contenedor tipo \lstinline|lista|, este cambia.
Esta es una
característica distintiva de una lista: es un contenedor de datos
\emph{mutable}.

\begin{itemize}
\item El constructor \lstinline|list()| nos permite crear una copia exacta de
una lista.
\begin{lstlisting}
L=[1,2,3,4,5]
LL=list(L)
L.append(12)
L,LL
\end{lstlisting}
\begin{Output}
	([1, 2, 3, 4, 5, 12], [1, 2, 3, 4, 5])
\end{Output}

\item Por ejemplo,  \lstinline|L.sort()| cambia la lista \lstinline$L$ de 
n\'umeros
enteros por la lista ordenada pero mantiene el nombre \lstinline$L$. Si m\'as 
adelante, en
nuestro programa,  necesitamos usar otra vez la lista \lstinline$L$ original es 
preciso
generar una copia mediante \lstinline|LL=list(L)|, o bien primero generar la
copia \lstinline$LL$ y ordenar \lstinline$LL$ manteniendo \lstinline$L$ con su 
valor original.

En particular, una asignaci\'on como \lstinline|LL = L.sort()| no crea la lista
\lstinline$LL$ aunque s\'{\i} ordena \lstinline$L$.

\item En general, si se \emph{alimenta} de cualquier otro contenedor iterable,
se crea una
lista con sus elementos. El siguiente ejemplo, que usa una cadena de caracteres,
se
entenderá mejor al leer la siguiente sección. Lo incluimos para ilustrar el uso
de otro
tipo de
datos iterable:
\begin{lstlisting}
Letras=list('ABCDE')
Letras
\end{lstlisting}
\begin{Output}
	['A', 'B', 'C', 'D', 'E']
\end{Output}


\item La función \lstinline|srange()| permite crear listas de
\hyperref[enteros]{enteros de
{\sage}.}
En python se usa la función \lstinline|range()|, que devuelve
\hyperref[enteros]{enteros de Python,}
a los que
no son aplicables los métodos de {\sage}. 
\begin{itemize}
 \item \lstinline|srange(j,k,d)|: devuelve los números entre \lstinline|j|
(inclusive) y
\lstinline|k|
(exclusive), pero contando de \lstinline|d| en \lstinline|d| elementos. A~pesar
de que el
uso más extendido es con números enteros, los números \lstinline|j, k|  y
\lstinline|d|
pueden ser enteros o no.
\end{itemize}
Abreviaturas:
\begin{itemize}
 \item \lstinline|srange(k)|: devuelve \lstinline|srange(0,k,1)|. Si, en
particular,
\lstinline|k|
es un natural, devuelve los naturales entre \lstinline|0| (inclusive) y
\lstinline|k|
(exclusive); y si \lstinline|k| es
negativo, devuelve una lista vacía.
 \item \lstinline|srange(j,k)|: devuelve la lista \lstinline|srange(j,k,1)|. Si,
en
particular, \lstinline|j| y \lstinline|k| son enteros, devuelve los enteros
entre 
\lstinline|j| (inclusive) hasta el anterior a \lstinline|k|. 
\item \lstinline|[a..b]| es equivalente a  \lstinline|srange(a,b+1)|.
\end{itemize}

\item El método \lstinline|.reverse()| cambia la lista por la resultante de
reordenar sus elementos dándoles completamente la vuelta.
\begin{lstlisting}
L=[2,3,5,7,11]
LL=list(L)
L.reverse()
L, LL
\end{lstlisting}
\begin{Output}
	([11, 7, 5, 3, 2], [2, 3, 5, 7, 11])
\end{Output}

\item Con \lstinline|.append()| se añade un nuevo elemento, y solo uno, al final
de la
lista. Como método, cambia la lista. Así, \lstinline|L.append(5)| equivale a
\lstinline|L=L+[5]| (o \lstinline|L+=[5]|). Para ampliar con más de un elemento,
existe
el método \lstinline|.extend()|.

\item Si los elementos de la lista son comparables, se pueden ordenar, de menor
a mayor,
con el método \lstinline|.sort()|.

\item \lstinline|sum|{\tt(<lista\_numérica>)} \footnote{Esta notación, que
usaremos
frecuentemente, indica el tipo de objetos a los que podemos aplicar la función
\lstinline|sum()|, en este caso  a los objetos de tipo {\tt lista de números} y
solo
a tales listas. Cuando la aplicamos, por ejemplo \lstinline|sum([1,2,3])|, no
escribimos los angulitos sino solo una lista, \lstinline|[1,2,3]| o el nombre,
por
ejemplo \lstinline|L|, de una lista de números.} calcula la suma de los
elementos de
la lista, que deben ser números. 


\item Listamos, para finalizar, métodos referentes a un elemento concreto y una
lista: el número de apariciones se averigua con el método \lstinline|.count()|;
información sobre la primera aparición la ofrece \lstinline|.index()|; 
se puede borrar con métodos como \lstinline|.pop()| o \lstinline|.remove()|; 
y se añade en una posición determinada con \lstinline|.insert()|. La ayuda
interactiva, 
\lstinline|(| tras el nombre del método y \texttt{tabulador}, nos amplia esta
información.

\end{itemize}
\end{enumerate}

\subsection{Otras listas}
Muchos métodos y funciones de sage devuelven listas como producto final de su
evaluación,
o, al menos, información fácilmente utilizable para generar una lista.
Terminamos esta
sección con varios de estos ejemplos.
\begin{itemize}
\item {\itshape List comprehension.} Más que un método o función, es una manera
abreviada
de creación de listas. Por su especial sintaxis, en ocasiones es 
difícil de entender en una primera lectura. En general, se utiliza para generar
listas
aplicando cierta transformación sobre otra lista, o con los elementos de otro
contenedor
iterable. Así, por ejemplo
\begin{lstlisting}
[ j^2 for j in [1..5] ]
\end{lstlisting}
tomará, uno a uno, los elementos de la lista \lstinline|[1,2,3,4,5]| y generará
la lista
de sus cuadrados:\\
\begin{Output}
	[1, 4, 9, 16, 25]
\end{Output}
Se puede, también, filtrar los elementos del contenedor original. De nuevo nos
adelantamos a un material posterior, aunque es fácil entender el siguiente
ejemplo.
\begin{lstlisting}
[ j^2 for j in [1..20] if j.is_prime() ]
\end{lstlisting}
\begin{Output}
	[4, 9, 25, 49, 121, 169, 289, 361]
\end{Output}

Volveremos sobre este asunto en la subsecci\'on \ref{bucles2}.
\label{ent-sage}
 \item Dado un \hyperref[enteros]{entero de {\sage}}, \lstinline|k|, el método
\lstinline|k.digits()|
devuelve
una lista cuyos elementos son sus dígitos. Por ejemplo, \lstinline|123.digits()|
devuelve
la lista  \lstinline|[3,2,1]| (atención al orden). Por defecto los dígitos se
consideran
en la base $10$, y una lista \emph{sin ceros a la izquierda}. Otra base de
numeración se
indica como primer argumento, y un número de dígitos fijo, \lstinline|m|, se
indica
asignando al parámetro \lstinline|padto| dicho valor: \lstinline|padto=m|.
\begin{lstlisting}
L=123.digits(padto=5)
LL=111.digits(2,padto=7)
L.reverse(),LL.reverse()
L, LL
\end{lstlisting}
\begin{Output}
	([0, 0, 1, 2, 3], [0, 1, 1, 0, 1, 1, 1, 1])
\end{Output}

\item Los enteros de {\sage} se pueden factorizar en primos con el método
\lstinline|.factor()|. El resultado no se muestra como una lista, pero en
ocasiones es
útil la lista de factores y potencias: el constructor \lstinline|list()| es
apropiado
para este objetivo.
\begin{lstlisting}
k=3888
print k.factor()
L=list(k.factor())
L
\end{lstlisting}
\begin{Output}
	2^4 * 3^5
	[(2, 4), (3, 5)]
\end{Output}
\end{itemize}



\subsection{Ejercicios con listas}
\label{ejer-0}
\begin{ejer}
Dada la lista \lstinline|L=[3,5,6,8,10,12]|, se pide:

\letra averiguar la posición del número $8$;

\letra cambiar el valor $8$ por $9$, sin reescribir toda la lista;

\letra intercambiar $5$ y $9$;

\letra intercambiar cada valor por el que ocupa su posición \emph{simétrica}, es
decir,
el primero con el último, el segundo con el penúltimo, ...

\letra crear la lista resultante de concatenar a la original, el resultado del
apartado
anterior.

\end{ejer}

\begin{ejer}
La orden \lstinline|prime_range(100,1001)| genera la lista de los números primos
entre
$100$ y $1000$. 

Se pide, siendo \verb|primos|\lstinline|=prime_range(100,1001)|:

\letra averiguar el primo que ocupa la posición central en \verb|primos|;

\letra averiguar la posición, en \verb|primos|, de $331$ y $631$;

\letra extraer, conociendo las posiciones anteriores, la sublista de primos
entre $331$ y
$631$ (ambos incluidos);

\letra extraer una sublista de primos que olvide los dos centrales de cada
cuatro;

\letra (**) extraer una sublista de primos que olvide el tercero de cada tres.
\end{ejer}


   
\section{Tuplas}
Las ``tuplas'' son similares a las listas, la mayor diferencia es que no podemos
asignar nuevo valor a los elementos de
una tupla. Decimos que las tuplas son ``inmutables''. Las listas no lo son
porque
podemos reasignar el valor de una
entrada de la lista, \lstinline|L[1] = 7|, y también un método puede modificar
el valor de una lista, por ejemplo, \lstinline|L.reverse()|
cambia el valor de~\lstinline|L|.


\begin{itemize}
\item Definimos una tupla mediante \lstinline|t = (1,2,3,4,5,6)|.
\item  Nos referimos a los elementos de una tupla con la notación \emph{slice},
de la
misma forma que a los de una lista, así: \lstinline|t[0]| es el elemento en la
primera
posición de la tupla, \lstinline|1|; \lstinline|t[1:5]| es la tupla
\lstinline|(2,3,4,5)|; etc.
\item Con el símbolo de la suma se concatenan tuplas: \lstinline|t + (1,5)| es
la tupla
\lstinline|(1,2,3,4,5,6,1,5)|. Y con el del producto, se repite:
\lstinline|2*(1,5)|
devuelve la tupla \lstinline|(1,5,1,5)|.

\item A diferencia de las listas, no se puede cambiar un elemento
de una tupla intentando reasignar su valor:
\begin{lstlisting}
t=(1,2,3,4,5,6,7)
t[1]=3
\end{lstlisting}
\begin{Output}
	TypeError: 'tuple' object does not support item assignment
\end{Output}
Si queremos cambiar algún elemento de una tupla, tenemos que reasignar toda la
tupla. 

En ocasiones, esta será una tarea ingrata y preferiremos un camino más directo.
Pensemos
en una tupla con algunas decenas de elementos.
Para el intento anterior, podríamos hacerlo con una sintaxis algo enrevesada:
\begin{lstlisting}
t1,t2=t[:1],t[2:] ##cortamos la tupla en dos, 
##evitando el elemento a sustituir
t1+(3,)+t2 	  ## concatenamos intercalando la tupla $(3,)$
\end{lstlisting}
\begin{Output}
	(1, 3, 3, 4, 5, 6, 7)
\end{Output}
Es fácil comprobar que \lstinline|(3)| no se interpreta como una tupla, pero
\lstinline|(3,)| sí.

Otra opción es cambiar a un contenedor tipo lista, más manipulable, realizar
los cambios y asignar el resultado a una tupla. Los constructores
\lstinline|list()| y
\lstinline|tuple()| son idóneos para este camino que se deja como ejercicio al
lector.

\item La función \lstinline|len()|, aplicada a una tupla, nos devuelve su
tamaño.

\item Los métodos \lstinline|.index()| y \lstinline|.count()| se aplican de
manera
análoga al caso de listas.

\end{itemize}



\subsection{Ejemplos con tuplas}
\label{factoriz}
\begin{ejem}
La composición \lstinline|list(factor())|, aplicada a un natural, devuelve una
lista de pares
\begin{lstlisting}
factores=list(factor(600)); factores
\end{lstlisting}
\begin{Output}
	[(2, 3), (3, 1), (5, 2)]
\end{Output}
Se puede iterar sobre esta lista para extraer los pares 
\begin{lstlisting}
for k in factores: print k,
\end{lstlisting}
\begin{Output}
	(2, 3) (3, 1) (5, 2)
\end{Output}
o \emph{desempaquetar} cada par, es decir extraer simultáneamente cada miembro
de la 
pareja al iterar la lista
\begin{lstlisting}
for a,b in factores: print a^b,
\end{lstlisting}
\begin{Output}
	8 3 25
\end{Output}
\end{ejem}

\begin{ejem}
La función \lstinline|xgcd()| aplicada a dos enteros, \lstinline|a,b|, 
devuelve una tupla con $3$ valores: \lstinline|(d,u,v)|. El primero es el máximo
común
divisor\footnote{\emph{gcd} son las siglas de \emph{greatest common divisor}. La
función
\lstinline|gcd()| devuelve, sin más, el máximo común divisor; \lstinline|xgcd()|
es una
versión e{\bf x}tendida.}, y se verifica la \emph{identidad de Bézout}:
$d=a\cdot u+b\cdot
v$.
\begin{lstlisting}
a,b=200,120
d,u,v=xgcd(a,b)
d,u,v,d-(a*u+b*v)
\end{lstlisting}
\begin{Output}
	(40, -1, 2, 0)
\end{Output}
\end{ejem}



En este curso usaremos mucho m\'as listas que tuplas, ya que, esencialmente,
realizan la misma funci\'on y las listas son m\'as flexibles. 

\section{Cadenas de caracteres}

Una cadena de caracteres es otro contenedor de datos inmutable, como las tuplas.
Lo que contiene son caracteres de un alfabeto, por ejemplo el nuestro ({\itshape
alfabeto latino}).

Dado que la {\sc criptograf\'{\i}a}  trata del enmascaramiento sistem\'atico de
un texto hasta hacerlo ilegible, pero recuperable conociendo la clave, es claro
que en el cap\'{\i}tulo \ref{cript}, dedicado a ella usaremos sistem\'aticamente
las funciones y m\'etodos que se aplican a  cadenas de caracteres.

\begin{itemize}
   
 \item Una {\tt cadena de caracteres} se forma
concatenando caracteres de un alfabeto, habitualmente formado por letras,
dígitos, y otros símbolos como los de puntuación. 

\item En el siguiente ejemplo se asigna a \lstinline|C| una cadena
\begin{lstlisting}[numbers=none]
C='Esta es nuestra casa.'
\end{lstlisting}
Las comillas delimitan el inicio y el final de la cadena. En este
caso es suficiente con comillas simples.
Si se quieren incluir comillas simples entre los caracteres, se usarán comillas
dobles\footnote{Comillas dobles no son dos simples consecutivas, sino las que
suelen
estar en la tecla \boton{2} de un teclado.} o triples (tres sencillas
consecutivas) para
delimitar; forzosamente triples si han de incluir simples y dobles. 


\item  \lstinline|str(k)| convierte el entero $k$ en una cadena de caracteres,
de
forma que \lstinline|str(123)| da como resultado la cadena {\lstinline|'123'|}. 

\item Las operaciones \lstinline|+| (concatenación) y \lstinline|*| (repetición)
se
utilizan y comportan como en el caso de listas.

\item  \lstinline|len(<cadena>)| es el número de caracteres de la cadena.

\item Para el acceso a subcadenas, se utiliza la notación \emph{slice}.
\begin{lstlisting}
frase='Recortando letras de una frase'
print (frase[3],frase[1:5:2],frase[6:10],frase[11:],frase[:4],frase[::3])
\end{lstlisting}
\begin{Output}
	('o', 'eo', 'ando', 'letras de una frase', 'Reco', 'Roaoeadu a')
\end{Output}

\item \lstinline|<cadena>.count(<cadena1>)| devuelve el número de veces que la
\lstinline|cadena1|  aparece ``textualmente'' dentro de la \lstinline|cadena|.
\begin{lstlisting}
frase='Contando letras de una frase'
frase.count('e')
\end{lstlisting}
\begin{Output}
	3
\end{Output}

\item \lstinline|<cadena>.index(<cadena1>)| devuelve el lugar en  que la
\lstinline|cadena1|  aparece \textsc{por primera vez} dentro de la
\lstinline|cadena|.
\begin{lstlisting}
frase='Buscando letras en una frase'
frase.index('a')
\end{lstlisting}
\begin{Output}
	4
\end{Output}

\item El método \lstinline|.split()| trocea una cadena de caracteres,
devolviendo una
lista con las subcadenas resultantes. Si no hay argumento, se
utiliza, por defecto, el espacio en blanco para trocear, eliminándose de la
lista de
subcadenas resultante. Si se indica al método una subcadena, se utiliza esta
para
recortar.
\begin{lstlisting}[showstringspaces=true]
frase='Una frase para trocear. 
Por defecto, se recorta por el espacio en blanco.'
print frase.split(' ')==frase.split()
palabras=frase.split(' ')
subfrases=frase.split('.')
palabras; subfrases
\end{lstlisting}
\begin{Output}
	True
	['Una', 'frase', 'para', 'trocear.', 'Por', 'defecto,', 'se',
'recorta', 
	'por', 'el', 'espacio', 'en', 'blanco.']
	['Una frase para trocear', ' Por defecto, se recorta por el espacio en
	blanco', '']
\end{Output}

\item El efecto inverso al uso de \lstinline|.split()| viene dado por el método
\lstinline|.join()|, que actúa sobre cadenas y espera un contenedor de cadenas;
o la
función \lstinline|join()|, que actúa sobre un contenedor de cadenas. Un
ejemplo,
conectado con el anterior, nos sirve para entender el uso
\begin{lstlisting}
'-'.join(palabras); join(palabras,'|')
\end{lstlisting}
\begin{Output}
	'Una-frase-para-trocear.-Por-defecto,
-se-recorta-por-el-espacio-en-blanco.'
	'Una|frase|para|trocear.|Por|defecto,
|se|recorta|por|el|espacio|en|blanco.'
\end{Output}
De no especificarse el segundo argumento de la función \lstinline|join()|, se
utilizará
un espacio en blanco simple como \emph{pegamento}. Para concatenar cadenas,
aparte del
signo \lstinline|+|, podemos usar \lstinline|join| con la cadena \emph{vacía}
\lstinline|''|
\begin{lstlisting}
palabras[0]+palabras[1]; ''.join(palabras); join(palabras,'') 
\end{lstlisting}
\begin{Output}
	'Unafrase'
	'Unafraseparatrocear.Pordefecto,serecortaporelespacioenblanco.'
	'Unafraseparatrocear.Pordefecto,serecortaporelespacioenblanco.'
\end{Output}

\item El método \lstinline|.find()| devuelve el primer índice en que aparece
una subcadena en la cadena a que se aplica; si no aparece, devuelve $-1$. Se
puede
delimitar el inicio de la búsqueda, o el inicio y el final (de omitirse este, es
el
último índice de la cadena). En la cadena \verb|frase| anterior aparece la
subcadena
\lstinline|'or'| en $3$ ocasiones. El siguiente código nos encuentra dónde:
\begin{lstlisting}
L=len(frase)
a=frase.find('or')
b=frase.find('or',a+1,-2)
c=frase.find('or',b+1)
a, b, c, frase.find('or',c+1)
\end{lstlisting}
\begin{Output}
	(25, 43, 49, -1)
\end{Output}

\end{itemize}

\noindent\emph{Sugerencia:} Pulsar el \verb|tabulador| tras escribir un punto
detr\'as
del nombre de una variable que contenga una cadena de caracteres, % 
\verb|<cadena>. + tabulador|, para encontrar los métodos aplicables
a cadenas de caracteres.


\subsection{Ejercicios}
\label{ejer-1}
\begin{ejer} Considérese el número $1000!$ (\lstinline|factorial(1000)|):
\begin{itemize}
 \item ¿en cuántos ceros acaba?
 \item ¿Se encuentra el número $666$ entre sus subcadenas? 
 En caso afirmativo, localizar (encontrar los índices de) todas las apariciones.
 \item Encontrar la subcadena más larga de doses consecutivos. Mostrarla con los
dos
d\'{\i}gitos que la rodean.
\end{itemize}
\end{ejer}

\begin{ejer}
Si se aplica la función \lstinline|sum()| a una lista numérica, nos devuelve la
suma de
todos sus elementos. En particular, la composición \lstinline|sum(k.digits())|,
para
\lstinline|k| un variable entera, nos devuelve la suma de sus dígitos (en base
$10$).
\begin{itemize}
 \item Calcular, con la composición \lstinline|sum(k.digits())|, la suma de los
dígitos
del número \lstinline|k=factorial(1000)|.
\item Calcular la misma suma sin utilizar el método \lstinline|.digits()|. 

\small
\emph{Sugerencia:} considerar la cadena \lstinline|digitos='0123456789'|, en la
que
\lstinline|digitos[0]='0'|, \lstinline|digitos[1]='1'|, ...,
\lstinline|digitos[9]='9'|.
Sumar los elementos de la lista
$$
\big[j*(\text{ veces que aparece } j \text{ en }1000!) :
\text{ con }j=1,2,3,4,5,6,7,8,9\big]\,.
$$
\normalsize
\end{itemize}

\end{ejer}


       
\section{Conjuntos}\hypertarget{conj}
Los conjuntos en SAGE se comportan y tienen, esencialmente, las mismas
operaciones que los conjuntos en Matemáticas.
La mayor ventaja que tienen sobre las listas es que, por su propia definición
como conjunto, suprimen las repeticiones
de elementos. Así \lstinline|[1, 2, 1]| tiene $3$ elementos (y están indexados),
pero
\lstinline|{1, 2, 1}| es lo mismo que \lstinline|{1,2}| y tiene solo $2$
elementos.


As\'{\i} podemos usar conjuntos para suprimir repeticiones en una lista:
transformamos
la lista en conjunto y el conjunto resultante en lista.

\begin{itemize}
\item Se utilizan las llaves para delimitar un conjunto:
\begin{lstlisting}
A = {1,2,1}
type(A), A
\end{lstlisting}
\begin{Output}
	(<type 'set'>, set([1, 2]))
\end{Output}
\item Aunque los conjuntos son mutables, sus elementos han de ser objetos
inmutables. Así, no podemos crear conjuntos con conjuntos o listas entre sus
elementos.

\small
\label{unhash}
\noindent\begin{minipage}{.45\textwidth}
\begin{lstlisting}
A={{1,2},{3}}
\end{lstlisting}
\begin{Output}
TypeError: unhashable type: 'set'
\end{Output}
\end{minipage}\vrule\hfill
\begin{minipage}{.45\textwidth}
\begin{lstlisting}
A={[1,2]}
\end{lstlisting}
\begin{Output}
TypeError: unhashable type: 'list'
\end{Output} 
\end{minipage}
\normalsize
\item Otra manera de crear conjuntos es con el constructor \lstinline|set()|,
que toma los elementos de cualquier contenedor
\begin{lstlisting}[columns=fullflexible]
l,t,s=[1,2,1],(3,2,3),'Hola gente'
set(l); set(t); set(s)
\end{lstlisting}
\begin{Output}
	set([1, 2])
	set([2, 3])
	set(['a', ' ', 'e', 'g', 'H', 'l', 'o', 'n', 't'])
\end{Output}
{\tt Nota:} La orden \lstinline|A=set()| crea un conjunto sin elementos, el
\emph{conjunto vacío}.

\item Los elementos de un conjunto no están ordenados ni indexados: si
\lstinline|A|
es un conjunto \lstinline|A[0]| no es nada.

\item  La instrucción \lstinline|1 in A| devuelve \lstinline|True| si el
\lstinline|1| es
uno de los elementos del conjunto \lstinline|A|.


\item Si \lstinline|A| y \lstinline|B| son dos conjuntos, la unión de dos
conjuntos se
obtiene con \lstinline:A | B:, la intersección
con \lstinline|A & B| y la diferencia con \lstinline|A-B|.
La comparación \lstinline|A<=B| devuelve
\lstinline|True| si todos los elementos de \lstinline|A| están en \lstinline|B|.

\item  Como para los contenederos ya estudiados, \lstinline|len(A)| es el número
de
elementos de un conjunto.

\item  Añadimos un elemento, por ejemplo el \lstinline|1|, a un conjunto
mediante
\lstinline|A.add(1)|, y lo suprimimos con \lstinline|A.remove(1)|.
Si se quieren añadir más elementos, contenidos en cualquier otro contenedor,
basta
aplicar el método \lstinline|.update()|
\begin{lstlisting}
A={-1,-2}
l=srange(100)
A.update(l)
len(A)
\end{lstlisting}
\begin{Output}
	102
\end{Output}
                         
\end{itemize}
  
\noindent\emph{Sugerencia:} Averiguar el uso de otros métodos como:
\lstinline|.pop()|,
\lstinline|.union()|, 

\lstinline|.intersection()|, \dots


\subsection{Ejercicios}

\begin{ejer}
A partir de la cadena de caracteres
\begin{lstlisting}
texto='Omnes homines, qui sese student praestare ceteris animalibus, 
summa ope niti decet ne uitam silentio transeant ueluti pecora, quae 
natura prona atque uentri oboedientia finxit.'
\end{lstlisting}
extraer la lista de los caracteres del alfabeto utilizados, sin repeticiones,
sin
distinguir mayúsculas de minúsculas y ordenada alfabéticamente.

\noindent\emph{Sugerencia:} La composición \lstinline|list(set())| aplicada a
una lista,
genera una lista con los elementos de la original sin repeticiones, ¿por qué?

\end{ejer}

\begin{ejer} {\bf Máximo común divisor.} 

Sin utilizar los métodos \lstinline|.divisors()| ni la función
\lstinline|max()|, 
elabora código que, a partir de dos números $a$ y $b$, calcule:
\begin{itemize}
\item El conjunto $\mathrm{Div}_a=\{k \in \mathbb N : k|a\}$
\item El conjunto $\mathrm{Div}_b=\{k \in \mathbb N : k|b\}$
\item El conjunto $\mathrm{Div}_{a,b}=\{k \in \mathbb N : k|a \text{ y } k|b\}$.
\end{itemize}
Una vez se tenga el conjunto de los \emph{divisores comunes}, encontrar el mayor
de ellos.
\end{ejer}


\begin{ejer} {\bf Mínimo común múltiplo.}

El mínimo común múltiplo, $m$, de dos números, $a$ y $b$, es menor o igual que
su
producto: $m\le a\cdot b$. Sin utilizar la función \lstinline|min()|, elabora
código 
que, a partir de dos números $a$ y $b$, calcule:
\begin{itemize}
\item El conjunto $\mathrm{Mult}_{a(b)}=\{k\in\mathbb{N}: a|k \text{ y }k<a\cdot
b\}$
\item El conjunto $\mathrm{Mult}_{b(a)}=\{k\in\mathbb{N}: b|k \text{ y }k<a\cdot
b\}$
\item El conjunto $\mathrm{Mult}_{a,b}=\{k\in\mathbb{N}: a|k\,,\,b|k \text{ y
}k<a\cdot
b\}$
\end{itemize}
Una vez se tenga este subconjunto de los \emph{múltiplos comunes}, encontrar el
menor de
ellos.
\end{ejer}

\begin{ejer}
Dada una lista de números enteros, construye un conjunto con los factores primos
de 
todos los números de la lista.

Indicación: Usa \lstinline|list(k.factor())| para obtener los factores primos.
\end{ejer}

Estos ejercicios, y los de las p\'aginas \pageref{ejer-0} y \pageref{ejer-1},
son importantes porque
vamos a estar usando  manipulaciones de estructuras de datos, sobre todo listas
y cadenas de caracteres, durante todo el curso. 
Soluciones en la hoja de {\sage}
\href{http://sage.mat.uam.es:8888/home/pub/??/}{\tt31-ESTRD-ejercicios-sol.sws},
                                
\section{Diccionarios}

Los elementos de una lista \lstinline|L| están indexados por los enteros entre
\lstinline|0| y \lstinline|len(L)-1| y podemos localizar cualquier elemento
si conocemos su orden en la lista. Los diccionarios son parecidos a las listas
en que los elementos están indexados, pero el conjunto de 
índices es arbitrario y adaptado al problema que queremos resolver.
\begin{itemize}
\item  Definimos un diccionario mediante
\begin{lstlisting}
diccionario = {clave1:valor1, clave2:valor2, ...}
\end{lstlisting}
Las claves son los identificadores de las entradas del diccionario y, por tanto,
han de
ser todas diferentes. En general, la información que contiene el diccionario
reside en
los pares  \lstinline|clave:valor|. Por ejemplo, un
diccionario puede contener la información \verb|usuario:contraseña|
   para cada uno de los usuarios de una red de ordenadores.

\item  Una vez creado el diccionario, recuperamos la información sobre el
valor correspondiente a una clave con la instrucción
\lstinline|diccionario[clave]|.
\begin{lstlisting}
granja={'vacas':3, 'gallinas':10, 'ovejas':112}
granja['ovejas']
\end{lstlisting}
\begin{Output}
	112
\end{Output}

\item  Definimos una nueva entrada, o cambiamos su valor, mediante
\begin{center}
\lstinline|diccionario[clave]=valor|,
\end{center}
 \noindent y suprimimos una entrada con
\lstinline|del diccionario[clave]|.
\begin{lstlisting}
granja['vacas']+=1 ## Se adquieren una vaca...
granja['caballos']=2 ## ... y dos caballos.
del granja['gallinas'] ## Se venden todas las gallinas.
show(granja)
\end{lstlisting}
\begin{Output}
	$\left\{\verb|vacas| : 4, \verb|ovejas| : 112, \verb|caballos| :
	2\right\}$
\end{Output}

\item Podemos crear un diccionario vacío con el constructor \lstinline|dict()|.
Si
pasamos al constructor una \emph{lista de pares}, este creará un diccionario con
claves
los primeros
objetos de cada par, y valores los segundos.

\begin{lstlisting}
Cuadrados_y_cubos=dict([(j,(j^2,j^3)) for j in [1..6]])
show(Cuadrados_y_cubos)
\end{lstlisting}
\begin{Output}
$\left\{1 : \left(1, 1\right), 2 : \left(4, 8\right), 3 : \left(9,
	27\right), 4 : \left(16, 64\right), 5 : \left(25, 125\right), 6 :
	\left(36, 216\right)\right\}$
\end{Output}


{\tt Nota:} Un buen constructor de listas de pares es \lstinline|zip()|. Si se
tienen dos
listas, \lstinline|L1| y \lstinline|L2|, la lista \lstinline|zip(L1,L2)| está
formada
por las parejas \lstinline|(L1[j],L2[j])| para \lstinline|j=0,1,...,| el menor
índice
final posible:
\begin{lstlisting}
L1, L2=[1..15], [10..16]
DD=dict(zip(L1,L2))
print DD
\end{lstlisting}
\begin{Output}
  	{1: 10, 2: 11, 3: 12, 4: 13, 5: 14, 6: 15, 7: 16}
\end{Output}

\item Los métodos \lstinline|.keys()| y \lstinline|.values()|
producen listas con las claves y los valores, respectivamente, en el
diccionario.
El método \lstinline|.items()| devuelve la lista de pares
\lstinline|(clave,valor)|.

\item La instrucción \lstinline| x in diccionario| devuelve \lstinline|True| si
\lstinline|x| es una de las claves del diccionario. 


\end{itemize}


 


Los diccionarios sirven para {\sc estructurar la informaci\'on}, haci\'endola
mucho m\'as accesible.  En el cap\'itulo \ref{cript}, dedicado a la
criptograf\'{\i}a,  veremos un ejemplo muy
pr\'oximo a la noci\'on de ``diccionario como un libro con palabras ordenadas
alfab\'eticamente'': queremos decidir de forma autom\'atica si un texto dado
pertenece a un idioma y disponemos de un archivo que contiene unas $120.000$
palabras en ese idioma,  una  en cada l\'{\i}nea. 

Podr\'{\i}amos crear una lista con entradas las palabras del archivo, pero es
mucho m\'as eficiente estructurar la informaci\'on como un diccionario.  Buscar
una palabra en una tal lista requiere recorrer la lista hasta que la
encontremos, y ser\'ia equivalente a buscar una palabra en un diccionario de
papel compar\'andola con cada palabra del diccionario empezando por la primera. 



En lugar de usar una lista, 
creamos un diccionario con claves tripletas de caracteres y cada tripleta toma
como valor la lista de todas las palabras en el archivo que comienzan
exactamente por esa tripleta.  De esta forma conseguimos que los valores en el
diccionario sean listas de longitud moderada, y resulta mucho m\'as f\'acil 
averiguar si una cierta palabra est\'a o no en el diccionario. 

\label{diccionario}
Puedes consultar la hoja de {\sage}
\href{http://sage.mat.uam.es:8888/home/pub/4/}{\tt 41-PROGR-diccionario.sws} para ver una implementaci\'on de esta idea, 
aunque {\sc es necesario haber visto el cap\'{\i}tulo siguiente}, dedicado a la
programaci\'on, para entenderla. 

Adem\'as, se discute en esa hoja la ventaja que se obtiene al usar diccionarios,
siempre que las longitudes de los valores est\'en equilibradas, frente a listas.
Este tipo de discusi\'on se repetir\'a a lo largo del curso, ya que repetidas
veces deberemos comparar, mediante {\itshape experimentos bien elegidos} los
m\'eritos o dem\'eritos de diversos m\'etodos para realizar los mismos
c\'alculos.
                                                                    
\section{Conversiones}
En ocasiones necesitamos convertir los datos contenidos en una cierta
estructura a otra, debido, esencialmente, a que la nueva estructura admite
m\'etodos que se adaptan mejor a las manipulaciones con los datos que
pretendemos hacer.
\begin{enumerate}
\item {\sc Tupla a lista o conjunto:} Para una tupla {\tt T}

\begin{lstlisting}
list(T)
set(T)
\end{lstlisting}

    
\item {\sc Lista a tupla:} \lstinline|tuple(L)|.

\item {\sc Cadena a lista:}
\begin{lstlisting}
C = 'abc'
list(C)
\end{lstlisting}
 \begin{Output}
['a', 'b', 'c']
 \end{Output}


 \item {\sc ?`Lista a cadena?}: 
 \begin{lstlisting}
C = 'abc'
str(list(C))
 \end{lstlisting}
 \begin{Output}
'['a', 'b', 'c']'
  \end{Output}
   En general esto {\sc no} es lo que queremos.
 \item {\sc Lista a cadena:}
 \begin{lstlisting}
C = 'abc'
join(list(C),sep="")
\end{lstlisting}
\begin{Output}
'abc'
 \end{Output}
\item  {\sc Lista a conjunto:} para una lista $L$,  \lstinline|set(L)|. Suprime
repeticiones en la lista.
 \item {\sc Conjunto a lista:} para un conjunto $A$, \lstinline|list(conjunto)|.

 \item {\sc Diccionario a lista de pares:} Para un diccionerio $D$, 
\lstinline|D.items()|.

\item {\sc  Lista de pares a diccionario:} Este peque\~no programa, con un
\hyperref[for]{bucle \lstinline|for|}, realiza la conversi\'on. 

\begin{lstlisting}
 def convert_list_dict(L):
        dict = {}
        for item in L:
            dict[item[0]]=item[1]
        return dict
\end{lstlisting}
\item Dadas dos listas, {\tt L1} y {\tt L2}, de la misma longitud podemos formar
una lista de pares mediante \lstinline|zip(L1,L2)|, y
    transformar esta lista en diccionario mediante la funci\'pn del apartado
anterior.
 \end{enumerate}                                             

